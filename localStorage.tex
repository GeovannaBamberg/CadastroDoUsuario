informações do localStorage:

1° recursos para salvar dados no navegador, que persiste após o recarregamento da página ou ao fechar a aba;
2° capacidade maxima de 10mb;
3° os dados são salvos no computador do usuário;
4° estes dados não possuem tempo de expiração, mas podem ser removidos;
5° os dados ficam na aba application do dev tools;

informações do sessionStorage:

1° recurso similar ao localStorage;
2° Capacidade máxima de 5mb;
3° Os dados são salvos no computador do usuário;
4° Os dados expiram quando a aba é fechada;


1 -INSERINDO DADOS:

localStorage.setItem("name", "matheus")

detalhe: independe do item que estivemos salvando, ele será salvo como string

2- restart sem perder dados

3 - RESGATAR ITEM 

const name = localStorage.getItem("nome");
alert (nome)

4- resgate de item que não existe

const lastName = localStorage.getItem("lastName")

5- remover item
localStorage.removeItem("nome")

6- limpar todos itens
localStorage.clear()

-------------------------
1 -INSERINDO DADOS:
sessionStorage.setItem('number',123)

2- reninciar e perder dados: 
(se atualizar mantem dados, mas se fechar e abrir não mantem )


3-remover item:
sessionStorage.removeItem("number")

4- limpar todos itens: 
sessionStorage.clear()

5 - RESGATAR ITEM 

const name = localStorage.getItem("nome");

---------------------------------------------------------------
Convertendo variaveis e objetos em json
const pessoa1 = {
    nome: "Geovanna",
    email:"geovanna.t.bamberg@gmail.com",
    senha: "Geo@35413590"
}

localStorage.setItem("persoa1", JSON.stringify(pessoa1))


const personObject = json.parse(getPerson)


sessionStorage.removeItem("number")

const getPerson = localStorage.getItem("pessoa1")

alert(getPerson)

alert(typeof personObject)
alert(personObject.job)